%%%%%%%%%%%%%%%%%%%%%%%%%%%%%%%%%%%%%%%%%
%	COMANDOS DE USO EN LATEX	%
%%%%%%%%%%%%%%%%%%%%%%%%%%%%%%%%%%%%%%%%

%  TIPOS DE DOCUMENTO
%  ******************

%  Artículo: \documentclass{article}
%  Libro:    \documentclass{book}
%  Carta:    \documentclass{letter}
%  Reporte:  \doculemntclass{report}

%  FORMATOS DE HOJA
%  ****************

%  Papel A4 : a4paper
%  Carta    : letterpaper
%  Papel A5 : a5paper
%  Papel B5 : b5paper
%  Ejecutivo: executivepaper
%  Legal    : legalpaper

%  TAMAÑO DE FUENTE
%  ********************

%  10pt, 11pt, 12pt

%\documentclass[12pt,a4paper]{article}

%  PAQUETES
%  ********

%  \usepackage{geometry}			% Para rehacer la geometría de la hoja
%  \usepackage{amsmath}				% Para hacer símbolos matemáticos
%  \usepackage{amssymb}				% Para hacer otro tipo de símbolos
%  \usepackage{graphicx}			% Para insertar figuras
%  \usepackage{wrapfig}				% Para insertar figuras dentro de un texto
%  \usepackage{array}				% Para hacer matrices
%  \usepackage{multirow}			% Para construir tablas
%  \usepackage{multicol}			% Para escribir en varias columnas
%  \usepackage{subcaption}			% Para captions de subfiguras
%  \usepackage[utf8]{inputenc}
%  \usepackage{lipsum}				% Para insertar texto random
%  \usepackage[usenames,dvipsnames]{color}  	% Para escribir con colores
%  \usepackage{color}
%  \usepackage{import}


%  TÍTULO
%  ******

%  \title{
%	  Mi primer documento \LaTeX
%  }

%  \author{
%		Bach. Luis Roberth Pérez Marcos %\\
		%\thanks{Universidad Nacional de Trujillo - 
		%\texttt{lperez@unitru.edu.pe}
		%}\\
		%Ph.D. Glenn C. Vidal Urquiza
		%\thanks{Polytechnic University of Puerto Rico, USA -
		%\texttt{gvidal@pupr.edu}
		%}\\
%   }

%  \date{26 de Eenero, 2023}

%\begin{document}

%  \maketitle

%  \renewcommand{\abstractname}{Resumen}

%  \begin{abstract}
%  \lipsum[1]
%  \end{abstract}

%  \newpage

%  \tableofcontents

%  \newpage

%  \section{Redacción de texto}
%  \label{Sección 1}
%  \section*{Mi primera sección}
%  \label{Sección 1}

%  \lipsum[2]

%       \subsection{Tamaño de letra}
     % \subsection*{Tamaño de letra}

% 	 \subsubsection{Tamaño tiny}
	%  \subsubsection*{Tamaño tiny}
% 	 {\tiny Este texto fue escrito con tamaño tiny}

% 	 \subsubsection{Tamaño scriptsize}
%  	{\scriptsize Este texto fue escrito con tamaño scriptsize}

%	  \subsubsection{Tamaño footnotesize}
%	  {\footnotesize Este texto fue escrito con tamaño footnotesize}

%	  \subsubsection{Tamaño small}
% 	 {\small Este texto fue escrito con tamaño small}

%	  \subsubsection{Tamaño normalsize}
%	  {\normalsize Este texto fue escrito con tamaño normalsize}

%	  \subsubsection{large}
% 	 {\large Este texto fue escrito con tamaño large}

% 	 \subsubsection{Tamaño Large}
%	  {\Large Texto con tamaño Large}

% 	 \subsubsection{Tamaño LARGE}
%	  {\LARGE Este es tamaño LARGE}

% 	 \subsubsection{Tamaño huge}
%	  {\huge Este tamaño}

%	  \subsubsection{Tamaño Huge}
% 	 {\Huge Este tamaño}

%	\vspace{1cm}

%     \subsection{Enfatizando texto}
	
%	\subsubsection{Negrita}
%	\textbf{Texto en negrita}

%	\subsubsection{Itálica}
%	\textit{Itálica o cursiva enfatizar}

%	\subsubsection{Tipeo}
%	\texttt{En formato de tipeo}

%	\subsubsection{En mayúsculas}
%	\uppercase{Mayúsculas}

%	\subsubsection{Subrayado}
%	\underline{Texto subrayado}

%	\subsubsection{A color}
%	Escribimos con rojo la palabra \textcolor{red}{documento}.

%	\subsubsection{Resaltado}
%	Resaltamos la palabra \colorbox{Orange}{documento}.

%	\subsubsection{Comando extra}
%	Para escribir textualmente un comando, como por ejemplo \verb+\textbf{}+

%     \subsection{Alineamiento de texto}

%	\subsubsection{Texto a la izquierda}
%	\begin{flushleft}
%	\lipsum[2]
%	\end{flushleft}

%	\subsubsection{Texto a la derecha}
%	\begin{flushright}
%	\lipsum[2]
%	\end{flushright}

%	\subsubsection{Texto al centro}
%	\begin{center}
%	\lipsum[2]
%	\end{center}

%	\subsubsection{ Otros comandos}
%	{\centering 
%	Oración al centro.

%	}
%	{\raggedright
%	Oración a la izquierda

%	}
%	{\raggedleft
%	Oración a la derecha

%	}

%	\subsection{A dos columnas}
%	\begin{multicols}{2}  %Tener activado el paquete multicol
%	\lipsum[2]
%	\end{multicols}

%	\subsubsection{Separar párrafos}

%	\lipsum[2]\\

%	\noindent  % Para borrar sangrías. Para hacer sangrías: \indent
%	\lipsum[2]

%     \subsection{Hacer listados}
%	\subsubsection{Ítems}
%	\begin{itemize}
%	\item Primera oración.
%	\item Segunda oración.
%	\item Tercera oración.
%	\end{itemize}
	
%	\subsubsection{Enumeración}
%	\begin{enumerate}
%	\item Primer elemento.
%	\item Segundo elemento.
%	\item Tercer elemento.
%	\end{enumerate}


%  \section{Escritura de ecuaciones}

%	\subsection{Entorno \textit{equation}}
%	\begin{equation}
%	\label{Ecuación 1}
%	a+b=5
%	\end{equation}

%	\begin{equation*}
%	a+b=5
%	\end{equation*}

%	\subsection{Ecuación en texto}
%	En esta línea ingresamos la ecuación $a+b=5$.\\
	
%	\noindent
%	Teorema de Pitágoras: $a^{2}+b^2=c^2$ \\

%	\noindent
%	Proporciones:  $\frac{a}{b} = \dfrac{c}{d} = \sqrt{\dfrac{a \times c}{b \cdot d}}$\\

%	\noindent
%	Energía cinética: $E_{cin\acute etica}=\dfrac{1}{2} m v^2$\\

%	\noindent
%	Energía potencial: $E_{potencial}=mgh$\\

%	\noindent
%	Componente en el eje x de la ecuación de Navier-Stokes :\[
%	\rho \left( V_{x} \dfrac{ \partial V_x }{ \partial x } + V_{y} \dfrac{ \partial V_x }{ \partial y } + 
%             V_{z} \dfrac{ \partial  V_x }{ \partial z } + \dfrac{ \partial V_x }{ \partial t } \right) =
%	\dfrac{ \partial { \sigma } }{ \partial x } + \mu \Delta_2 V_x + \rho g_x
%	\]

%	\noindent
%	Definición de velocidad: $\vec{V}=\dfrac{d \vec{x}}{dt}$\\
 
%  \newpage

%  \section{Insertar imagen}

%	\subsection{Entorno \textit{figure}}

%	\lipsum[1]
%	\begin{figure} [h!]  % h: insertar aquí , b: al fondo , t: arriba, p: en siguiente página , ! : para forzar
%	\centering  % \raggedright  \raggedleft   \hspace{}
%	\includegraphics[scale=0.2]{Figuras/figura.png} % width=  , height =   , angle=  
%	\caption{Primera figura del documento}
%	\label{Figura 1}
%	\end{figure}

%	\newpage

%	\subsection{Entorno \textit{wrapfigure}}

%	\begin{wrapfigure}{r}{0.3\textwidth}
%	\includegraphics[scale=0.1]{Figuras/figura.png}
%	\caption{Segunda figura}
%	\end{wrapfigure}

%	\lipsum[1]

%  \section{Hacer referencias}

%  Para hacer referencias se usa el comando \verb+\ref{}+. Por ejemplo, con ese comando estamos haciendo
%referencia de la Sección \ref{Sección 1}, la Ecuación \ref{Ecuación 1} y la Figura \ref{Figura 1}. Muchos 
%artículos han sido escritos con latex.



%\end{document}
